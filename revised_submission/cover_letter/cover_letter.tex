% cover letter for GCA submission

\documentclass[10pt]{letter}

% Colors
\usepackage{fancyvrb}
\usepackage[table]{xcolor}
\definecolor{tableShade}{HTML}{F1F5FA}   %iTunes
\definecolor{tableShade2}{HTML}{ECF3FE} %Finder

\definecolor{cambridgedarkblue}{RGB}{0,62,114}
\definecolor{cambridgeblue}{RGB}{0,115,207}
\definecolor{cambridgelightblue}{RGB}{106,173,228}

\definecolor{cambridgedarkorange}{RGB}{200,78,0}
\definecolor{cambridgeorange}{RGB}{227,114,34}
\definecolor{cambridgelightorange}{RGB}{239,189,71}

\definecolor{cambridgedarkgreen}{RGB}{67,81,37}
\definecolor{cambridgegreen}{RGB}{88,166,24}
\definecolor{cambridgelightgreen}{RGB}{168,180,0}

\definecolor{cambridgedarkviolet}{RGB}{65,45,93}
\definecolor{cambridgeviolet}{RGB}{142,37,141}
\definecolor{cambridgelightviolet}{RGB}{181,147,155}

\definecolor{cambridgedarkazur}{RGB}{21,101,112}
\definecolor{cambridgeazur}{RGB}{0,179,190}
\definecolor{cambridgesportingblue}{RGB}{163,193,173}
\definecolor{Grey}{rgb}{0.5, 0.5, 0.5}
\newcommand{\grey}[1]{\textcolor{Grey}{#1}}

\RequirePackage{hyperref}
\hypersetup{pdftex, plainpages = false, pdfpagelabels, 
  pdfpagelayout = useoutlines,dvips, 
  pdfauthor={R. Myhill, myhill.bob@gmail.com},
  pdftitle={Cover letter},
  pdfsubject={Cover letter},
  pdfkeywords={},
  pdfproducer={pdfLaTeX with hyperref},
  pdfcreator={pdfLaTeX}
  bookmarks,
  bookmarksopen = false,
  bookmarksnumbered = true,
  breaklinks = true,
  linktocpage,
  colorlinks = true,
  linkcolor = cambridgeblue,
  urlcolor  = cambridgeviolet,
  citecolor = cambridgedarkorange,
  anchorcolor = cambridgedarkazur,
  hyperindex = true,
  hyperfigures}

\oddsidemargin=.2in
\evensidemargin=.2in
\textwidth=5.9in
\topmargin=-.5in
\textheight=9in


\name{Robert Myhill \\
School of Earth Sciences\\
University of Bristol\\
Wills Memorial Building\\
Queens Road\\
Bristol BS8 1RJ\\
United Kingdom \\
Tel: +44 (0) 117 954 5400 \\
\href{mailto:bob.myhill@bristol.ac.uk}{bob.myhill@bristol.ac.uk}}


\date{\today}

\begin{document}


\begin{letter}{}
                           
\opening{Dear editor,}

Please find attached our revised manuscript entitled \emph{Hydrous
  melting and partitioning in and above the mantle transition zone:
  insights from water-rich MgO-SiO$_2$-H$_2$O experiments}, written
for potential publication as an \emph{Article} in \emph{Geochimica et
  Cosmochimica Acta}.

We have addressed all of the comments made by the reviewers. 
I have also made some minor changes to the text to improve
readability, and simplified some of the analysis so that it can be
more easily reproduced by others. The simplification (using the
intermediate composition melts from de Koker et
al., rather than creating a solution model) has a
negligible effect on the figures and conclusions in the paper.


\textbf{Responses to editorial and reviewer comments}
\begin{Verbatim}[commandchars=+\[\]]
Line 28: could use a reference at the end of this sentence (e.g. Spiegelman).
+underline[Two references added]

Line 33: could use a reference at the end of sentence about water
budget 
(e.g. Hirschmann)
+underline[Reference added]

Line 60: I think you might want a better segue here into the approach
you take in this study. Something like "Here we use the approach of 
bracketing the liquidus surface to determine melt composition as a 
function of temperature in the system MSH…".
+underline[The end of the introduction has been rewritten, starting with this suggestion.]

I found the introduction to be well written but in the end it fails to
let the reader know exactly what the tack is and what the goals of
this paper are, and even what system you will be working in and why.  
You could leave the business about how to make starting materials 
with high water contents until the methods, and end the introduction 
by telling the reader that you will use the liquidus phase relations
in MSH to develop thermodynamic models for water solubility in melts 
and partitioning models, which you will apply to understanding hydrous 
melting at transition zone conditions.
+underline[I agree. I've rewritten this paragraph]

Lines 70-75 (end of the introduction): Again, I agree with Mike that
the end of the intro section is a bit abrupt and suddenly becomes too
technical about the choice of starting mix water source. Such
information should become part of the methods (starting mix)
section. 
The authors are better of ending the introduction by saying generally 
what they have done in this study.
+underline[Paragraph rewritten.]

Line 78: "Starting compositions in the system MSH…"  
It also might be good to provide some explanation of the rationale 
for choosing the starting compositions, i.e. "We chose compositions 
along the bounding binaries, and within the ternary in order to 
contrain phase relations…"
+underline[Good suggestion. Detail added.]

Line 105: Any idea about water loss in any of the experiments? 
How much time do you have as a function of temperature, or is 
this intuition from experience?
+underline[Added reference to Novella et al. (2016), which has more details]

Line 137: "…concentrated hydrous melts". Presumably you mean water-rich.
+underline[Changed to “water-bearing silicate melts”]

Line 193: typo, should be 'note' not 'notes'
+underline[Typo corrected]

Line 203: I am probably just being thick, but here and other places 
you talk about "the experimentally determined value". 
What value are you referring to and from what source?
+underline[These values are from Stixrude and Lithgow-Bertelloni (2011).]
+underline[I cite the study for periclase (line 190), but have added another]
+underline[reference to the paper here.]

 
Line 227: You might consider a new section here as currently the 
segue is a bit clunky.
+underline[Section split into endmember and solution model subsections]

Line 269: Should read "…between quench crystals and liquidus phases…."
+underline[Corrected]


Table 1:  It took me a little while to get my head around the way the 
results are tabulated in Table 1, but in the end I think it works. You 
may want to add in the table caption that melts are present in all run 
products. There is also a typo in the caption "en-dash" should just by 
"dash".
+underline[Done]

Line 276: The invariant is drawn at ~1250 C rather than 1200. 
Also, why call it a fluid here… I would stick with liquid. You may
also want to clarify the key either by adding e.g. Brucite-L 
(or maybe br-L), etc.
+underline[Text changed to 1250C rather than 1200. ]
+underline[Reference to fluid changed to liquid. I haven't changed the key, ]
+underline[as all legend entries implicitly include liquid. The addition to ]
+underline[the caption of Table 1 should be sufficient (although I can]
+underline[change this if you think I should).]

Line 304. I am not understanding the use of the term 'metastable'
here.  It may be a good idea as well to be clear that Fig. 4 is not a binary
join, but is a T-X section depicting equilibrium crystallization. 
Actually, the figure is not drawn correctly based on Fig. 7, as there 
should be a fields for fo-en-liq and en-liq…
+underline[This sentence has been rephrased. I've also added the low]
+underline[temperature en-bearing fields to Figure 7]

Line 312: "…explained by an MgO…"
+underline[Corrected!]

The concave up nature if the liquidii in this study is very
interesting and convincing. Is there any chance this in part reflects
H2O loss in higher water content experiments? I don't think so, 
but just a thought.
+underline[Interesting thought. Given the different temperatures at which the ]
+underline[liquidus curves turn over in the fo-H2O, en-H2O and stv-H2O systems, ]
+underline[I suspect not. Yamada et al. (2004) saw something very similar, and we ]
+underline[didn't see any evidence for melt loss. I've made a comment to this]
+underline[effect at the end of the MgSiO3-H2O subsection.]

Fig. 7. I don't like the way you label the T contours on the
sidelines. I think you can label them within the ternary as is
typically done, and then you should plot the phases that occur on the 
bounding binary joins (e.g. br, fo, en). The light grey points are not 
explained in he caption and are a distraction. I also found a few 
inconsistencies with the liquidus phase relations depicted and the 
results in Table 1, but it was not clear to me whether this diagram 
is based on your best guess or is made from the thermodynamic 
model.
+underline[Changed labelling]
+underline[Removed grey points]
+underline[Added description of how the figure was created]

 
Line 388. I am not sure about this sentence. What you say is true, 
but the amount of melt will always depend on the water content of 
the source, and so a rapid increase in mass fraction is a relative
thing. If the source contains a few hundred ppm water, then there 
will never be a lot of melt, even though the mass fraction may
increase from say 0.0001 to 0.01, it is still small and hard to
detect. 
Seismic detection will depend mostly on their alignment 
The change in composition and density is of course very important to
its mobility.
+underline[I've reworked these sentences to focus on the channelisation,]
+underline[which will be promoted by the increase in silicate solubility in the]
+underline[melt and increase melt fraction and connectivity.]

Lines 385-390: Similar to the comments of the reviewer, I also note 
that this argument is unlikely to be correct. The melt proportion is 
controlled by the bulk water content of the source and the D of water.
 So even if water content in melt changes significantly, that does not 
necessarily mean generation of seismically detectable melt fraction. 
+underline[See above]

Plus, a temperature change of 1300 to 1700 degree C is huge. Is it 
likely that such a large change in T is realized even for deeply 
subducted slab-mantle interface, which should already be thermally 
equilibrated more than in typical subduction zone conditions?
+underline[I've now changed the sentence to be more specific about temperature]
+underline[ranges in the mantle at ~400 km.]

Line 395. Need a ref here regarding channelization.
+underline[Reworded to link the two comments on channelization.]

Conclusion: This may simply be a writing style issue, but the 
discussion section seems to end somewhat abruptly. Would the 
paper benefit from a small concluding remarks section or something 
similar?
+underline[Agreed. I've added a concluding remarks section, as suggested.]

Line 476; Figure # should be Figure 9
+underline[Corrected]

Figure 9 and 10: In the figure captions write that these are partition 
coefficients for H2O. Of course you mentioned in the text but from 
the figures one can't tell its partition coefficient for what.
+underline[Done]

Figure 9 and 10 y-axes: Again, write H2O as subscript to the Ds 
so that the figures are self-explanatory.
+underline[Corrected]
\end{Verbatim}
\closing{Best wishes,}

\end{letter}

\end{document}
